\documentclass[]{article}

\usepackage{tabularx}
\usepackage[dutch]{babel}
\usepackage{amsmath}
\usepackage{graphicx}
\usepackage{amsmath}
\usepackage{epstopdf}
\usepackage[parfill]{parskip}

\newcommand{\opgave}[1]{\section*{Opgave #1}}

\begin{document}

\opgave{6}

De Arnoldi methode berekent iteratief de eigenwaarden van een matrix A. Dit gebeurt door een Hessenberg matrix te maken van A, die in iedere stap groter wordt. De eigenwaarden van die groeiende matrix, de zogenaamde ritz waarden, zijn een goede benadering voor de eigenwaarden van de matrix A. Alhoewel er in het begin veel minder ritz waarden zijn dan eigenwaarden van A, vormen de ritz waarden toch vrij snel goede benaderingen voor alle eigenwaarden. Dit is vooral effectief voor ijle matrices omdat de eigenwaarden daarvan dicht bij elkaar liggen. \'{E}\'{e}n ritz waarde is dan genoeg om meerdere eigenwaarden van A te benaderen. Alhoewel de benaderingen zeer ruw zijn met zo een klein aantal iteraties zien we toch dat we hier met maar 40 iteraties nog slechts een gemiddelde fout hebben van 10 $\%$. Dit is niet weinig, maar we hebben dit wel met weinig rekenwerk kunnen vinden. In figuur ~\ref{opgave6} zien we ook dat de 20 grootste eigenwaarden beter benaderd worden door de ritz waarden. Dit is omdat de ritz waarden sneller convergeren naar de uitschieters. 
\linebreak Er valt op te merken dat willekeurige $n\times n$ matrices \'{e}\'{e}n eigenwaarde hebben die ongeveer n keer de gemiddelde waarde van de elementen van die matrix is, terwijl alle andere eigenwaarden rond 0 gegroepeerd zijn. De Arnoldi methode zal heel snel convergeren naar die uitschieter, die meestal de meest interessante eigenwaarde is. 

\begin{figure}[h]
\noindent \includegraphics[width=1\linewidth]{Opgave6.eps}
\caption{Convergentie}
\label{opgave6}
\end{figure}


\end{document}